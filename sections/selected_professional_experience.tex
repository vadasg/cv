\section{\sc Professional Experience} %1


{\bf BAE Systems, Inc.}, 2014 - present
\begin{list1}
%\vspace{-4mm}
\item[] \textit{Senior Principal Scientist}
\item[] Drives research and development of image processing, image analysis, and image understanding solutions for multiple image and video sensor data applications, using techniques that span deep learning, signal/image processing, machine learning, artificial intelligence, computer vision, object recognition, and probabilistic tracking.  Works in a multidisciplinary team with customers and contractors to develop externally sponsored research programs and expand the technological capabilities of the company.
 
\end{list1}


{\bf Chatham University}, Pittsburgh, Pennsylvania; 2010 - 2014
\begin{list1}
%\vspace{1mm}
\item[] {\em  Assistant Professor of Physics}, Department of Physics
%\begin{list2}
%\item Developed general theory for optimal forcing of nonlinear dynamics, both continuous and discrete in time.  Validated theory numerically. 
%\item Designed and built world's first interreality physics experiment.  Developed virtual reality pendulum application and constructed laboratory experimental apparatus.
%\item[] Publications (see numbered references next page): \citebibpublications{Trout13EJC, GintautasKenyon11, Ham10BioCyber, Gintautas10JICRD} 
%\end{list2}
%\end{list2}
\end{list1}


{\bf Los Alamos National Laboratory}, Los Alamos, New Mexico; 2008 - 2010
\begin{list1}
%\vspace{1mm}
\item[] {\em Postdoctoral Research Associate}, Center for Nonlinear Studies, Theoretical Division
%\item[] {\em Visiting Graduate Student}, Summer 2007; January 2008 - August 2008
%    \begin{list2}
%\item Current work: developed information theoretic formalism for coordination in multi-agent searches.  Performs computer simulations and analysis.  Extends formalism to vision models in computational neuroscience.
%\item Helped develop information theoretic formalism for the reconstruction of functional subgraphs in complex networks.  Analyzed spiking time series from mouse cortical neuronal networks grown in vitro.  
%\item[] Publications: \citebibpublications{Gintautas09SBP, Rodriguez09FirstMonday, Gintautas09PE, Bettencourt08PRL, HamGintautasGross08MEA} 
%\end{list2}
\end{list1}

